\documentclass[11pt,a4paper]{article}
\usepackage[utf8]{inputenc}
\usepackage[ngerman]{babel}
\usepackage{csquotes}
\usepackage{hyperref}
\usepackage{tabularx}
\usepackage{calc}
\usepackage{geometry}
\usepackage[scaled]{helvet}
\usepackage[T1]{fontenc}

\renewcommand{\familydefault}{\sfdefault}

\geometry{
  left=2cm,
  right=2cm,
  top=2cm,
  bottom=2cm,
  includeheadfoot
}

% Schriftgrößen und Titeldefinitionen
\newcommand{\titleformat}{\fontsize{18pt}{22pt}\selectfont}
\newcommand{\sectionformat}{\fontsize{14pt}{18pt}\selectfont}
\newcommand{\subsectionformat}{\fontsize{11pt}{14pt}\selectfont}
\newcommand{\bodytextformat}{\fontsize{10pt}{12pt}\selectfont}


\begin{document}

\begin{flushleft}
    {\titleformat\textbf{Integration einer Sprachsteuerungsfunktion in Mobile Apps}}\\[0.5cm]
  \end{flushleft}

\bodytextformat

\noindent
\begin{tabularx}{\linewidth}{@{}p{4cm}X@{}}
    \textbf{Themenbereiche:} & Sprachsteuerung, Audio, Machine Learning, Triggerwort-Erkennung, Mobile Apps \\
    \textbf{Studierender:} & Ruben Nuñez \\
    \textbf{Dozent:} & Dr. Florian Herzog \\
    \textbf{Experte:} & Damien Piguet \\
    \textbf{Auftraggeber:} & Stefan Reinhard \\
    \textbf{Keywords:} & Sprachsteuerung, Mobile Apps, Machine Learning, Triggerwort-Erkennung, Audioverarbeitung, Datensatzerstellung, Modelltraining, App-Integration \\
\end{tabularx}
    




\section{Aufgabenstellung}
Die Bachelorarbeit zielt darauf ab, eine Sprachsteuerungsfunktion für mobile Apps zu entwickeln, wobei der Fokus auf der Erkennung von Triggerwörtern in akustischer Sprache liegt. Zentrale Aspekte sind die Entwicklung eines Machine Learning-Modells, seine Implementierung in eine mobile Plattform wie iOS und die Beachtung von Datenschutz und ethischen Richtlinien beim Erstellen und Verwenden eines Datensatzes.


\section{Ergebnisse}
Im Rahmen Bachelorarbeit wurden folgende Ergebnisse erzielt:

\begin{itemize}
    \item Ein \textbf{Machine Learning-Modell}, das in der Lage ist, Triggerwörter in akustischer Sprache zu erkennen.
    \item Eine \textbf{Integration in eine iOS-App}, die das Modell verwendet, um Triggerwörter in Echtzeit zu erkennen.
    \item Ein \textbf{Datensatz}, welcher ethische Richtlinien und Datenschutzbestimmungen einhält welcher für das Training des Modells verwendet wurde.
    \item Eine \textbf{Dokumentation}, die die Entwicklung des Modells, die Erstellung des Datensatzes und die Implementierung der App beschreibt.
  
\end{itemize}


\section{Lösungskonzept}


\section{Spezielle Herausforderungen}


\section{Ausblick}


\end{document}
