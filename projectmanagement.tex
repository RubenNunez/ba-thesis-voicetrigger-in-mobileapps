%! TEX root = main.tex  %
\documentclass[main.tex]{subfiles} % Wichtig!

\begin{document}


\subsection{Projektmanagement}

Wie im vorhergehenden Kapitel «Agile Entwicklung - Automatisiertes Testen / DevOps»
beschrieben, sind sämtliche gesammelten Informationen und gefällten Entscheidungen
in einem Artefakt zu vermerken. Anleitungen für typische Softwareartefakte können
dabei nützliche Wegleitungen sein.

\subsubsection{Produkt Backlog}

In der Vorbereitungsphase kann ein anfängliches Produkt Backlog als einfache Tabelle
dargestellt werden. Ein Beispiel für eine solche Tabelle ist in Abbildung 5 dargestellt.

\begin{figure}[h]
    \centering
    %\includegraphics[width=0.7\linewidth]{path_to_image} % Ersetzen Sie "path_to_image" durch den Pfad zu Ihrem Bild
    \caption{Tabelle für das anfängliche Product Backlog}
    \label{fig:backlog_table}
\end{figure}


\subsubsection{Risikomanagement}
Risikomanagement dient dem Zweck, mögliche Probleme vorwegzunehmen. Die Verwendung von
Checklisten, Brainstorming mit den Anspruchsgruppen und die von Erfahrungen
aus früheren Projekten sind mögliche Strategien zur Identifikation möglicher Risiken.

\begin{table}[h]
    \centering
    \caption{Beispiel-Tabelle für Risikomanagement}
    ... % Hier sollte der Tabelleninhalt definiert werden.
\end{table}



\end{document}

