\documentclass[11pt,a4paper]{article}
\usepackage[utf8]{inputenc}
\usepackage[ngerman]{babel}
\usepackage{csquotes}
\usepackage{graphicx}
\usepackage{geometry}
\usepackage{hyperref}
\usepackage{subfiles}
\usepackage[style=apa, backend=biber]{biblatex}
% \usepackage{indentfirst} 
% \usepackage[
%     colorlinks=true, 
%     linkcolor=blue,
%     urlcolor=blue,
%     citecolor=blue,
%     anchorcolor=blue
% ]{hyperref}

\addbibresource{references.bib}

\geometry{a4paper, total={170mm,257mm}, left=20mm, top=20mm}

\title{
    {\LARGE Bachelorarbeit}\\[2em]
    {\textbf{Integration einer Sprachsteuerungsfunktion {\break} in Mobile Apps}}
}
\author{Rubén Nuñez}
\date{Herbstsemester 2023}

\begin{document}

\maketitle
\thispagestyle{empty} %Keine Seitennummerierung
\newpage

\section*{Eidesstattliche Erklärung}
\newpage


\tableofcontents
\newpage


\section{Problem, Fragestellung, Vision}
Das Problem dieser Arbeit ist im wesentlichen die Erkennung von Triggerwörtern innerhalb
des Kontext einer App. Grundsätzlich ist es unüblich, dass mobile Apps eine
integrierte Sprachsteuerungsfunktion anbieten.
\newpage


\section{Stand der Forschung}


Laut \textcite{einstein1905} ist $E$ gleich $mc^2$. \\
\noindent Dieser Text wird ohne Einzug beginnen.
Wie \textcite{einstein1905} bemerkte, ist \(E=mc^2\). \\
\glqq Ein bekanntes Ergebnis aus der Relativitätstheorie ist \(E=mc^2\) (\cite{einstein1905}).\grqq



\section{Anhang}

\subfile{projectmanagement.tex}

\newpage
\section*{Aufgabenstellung}
Integration von Sprachsteuerungstechnologien in Mobile Apps, insbesondere zur Erkennung
von Triggerwörtern.

\section*{Projektteam}
\begin{itemize}
    \item Student:in: Rubén Nuñez
    \item Betreuer:in: Herzog
    \item Firma: Bitforge AG
\end{itemize}

\section*{Auftraggeber}
\begin{itemize}
    \item Firma: Bitforge AG
    \item Ansprechperson: Stefan Reinhard
    \item Funktion: Head of Mobile
    \item Adresse: Zeughausstrasse 39, 8004 Zürich
    \item Telefon: +41 55 211 02 41
    \item E-Mail: stefan.reinhard@bitforge.ch
    \item Website: www.bitforge.ch
\end{itemize}

\section*{Ausgangslage und Problemstellung}
Sprachsteuerungstechnologien haben ein grosses Potenzial und werden bisher vor allem als
Sprachsteuerungsassistenten genutzt. Während es etablierte Sprachassistenten wie Siri gibt,
fehlt es an Lösungen für eine integrierte Sprachsteuerung in Mobile Apps, insbesondere in
Bezug auf das Erkennen von Triggerwörtern.

\section*{Ziel der Arbeit und erwartete Resultate}
Ziel der Arbeit ist es zum einen, eine Grundlage zu schaffen, um ein Triggerwort oder eine
Sequenz von Triggerwörtern in der akustischen Sprache erkennen zu können. Dabei werden
Methoden und Werkzeuge aus dem Bereich des Machine Learnings verwendet. Zum anderen soll
diese Erkenntnis in eine mobile Plattform wie iOS oder Android integriert werden. Für den
Rahmen dieser Arbeit genügt die Integration in eine der genannten Plattformen. Weiterhin
werden das Thema Datenschutz und die ethischen Aspekte berücksichtigt.

\section*{Gewünschte Methoden, Vorgehen}
Das Projekt kann beispielsweise in drei Phasen durchgeführt werden: Technische Abklärungen,
Datensammlung und Modelltraining, sowie die Erarbeitung eines Prototypen. Agile
Vorgehensweisen sind wünschenswert.

\section*{Kreativität, Methoden, Innovation}
Bisher sind Sprachsteuerungsfunktionen fast ausschliesslich grossen Akteuren wie Siri
vorbehalten. Der innovative Ansatz dieser Arbeit zielt darauf ab, einen Anreiz zu setzen,
um diese Funktionen auch in herkömmlichen Apps einzusetzen. Die handfreie Bedienung durch
Sprachsteuerung hat das Potenzial, das Benutzererlebnis erheblich zu verbessern.

\section*{Sonstige Bemerkungen}
Grundkenntnisse in Machine Learning, speziell im Bereich der Spracherkennung, sowie
Erfahrung mit entsprechenden APIs sind erforderlich.



% Literaturverzeichnis
\newpage
\printbibliography[title=Literaturverzeichnis]



\end{document}
